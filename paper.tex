% Paper follows the following rules:
%   - 9pt font
%   - 10pt leading
%   - 0.33in column separation
%   - 3.33in textwidth
%   - 0.75in left, right, and top margins
%   - 1.00in bottom margin
%   - 1.00in bottom margin for the first column for ACM Copyright
%   - no page numbers
\documentclass[letterpaper,twocolumn]{article}
\usepackage{times}
\renewcommand{\normalsize}{\fontsize{9pt}{10pt}\selectfont}
\setlength{\columnsep}{0.33in}
\setlength{\textwidth}{3.33in}
\usepackage[margin=0.75in,top=0.75in,bottom=1.00in]{geometry}
\usepackage{fancyhdr}
\pagestyle{fancy}
\renewcommand{\headrulewidth}{0pt}
\renewcommand{\footrulewidth}{0pt}
\lfoot{\vspace{1in}}
\cfoot{}
\rfoot{}
\lhead{}
\chead{}
\rhead{}

\begin{document} % Start of the document
\date{} % No date
\title{Introducing Secure Software Development for High School Educators: A Project-based Learning Approach}
\author{
        {\rm Edward Dang}\\
        Worcester Polytechnic Institute\\
        Worcester, MA 01609 USA\\
        edang@wpi.edu\\

        {\rm Tabitha Senty}\\
        ISD 196\\
        Rosemount, MN 55068 USA\\
        tabitha.senty@apps.district196.org\\
        
        {\rm Brandyn Miller}\\
        Palmerton Area School District\\
        Palmerton, PA 18071 USA\\
        bmiller@palmerton.org\\
        
        {\rm Marti Shirley}\\
        Glenbard Township \#87\\
        Glen Ellyn, IL 60137 USA\\
        marti\_shirley@glenbard.org\\
        
        {\rm Pamela Dooley}\\
        Stride Private Academy\\
        Hearndon, VA 20170 USA\\
        pdooley@k12.com\\
        
        {\rm Marina Moshchenko}\\
        Brooklyn Technical HS\\
        Brooklyn, NY 11217 USA\\
        mmoshchenko@schools.nyc.gov\\

    }
\maketitle % Generate the title
\thispagestyle{fancy} % Use the fancy page style

\section*{Abstract} % The * removes the section number
This ACM conference publication presents a research study that introduces high school educators to secure software development, through a project-based learning approach. Led by a collegiate computer science student, the project focused on designing, implementing, and maintaining a school management system by rigorously following the Secure Software Development Life Cycle (SSDLC). This methodology integrates security considerations at every stage of software development, ensuring that potential vulnerabilities are identified and addressed early in the process, resulting in a secure and robust final product.
The study involved a comprehensive exploration of the technical aspects of the school management system, including user authentication, data encryption, access control, and logging mechanisms. By simulating various attack scenarios, the team uncovered vulnerabilities and implemented security measures to mitigate them. Additionally, the project aimed to empower high school educators with the knowledge and skills needed to recognize and prevent security threats, fostering a culture of security awareness within the educational environment.
The outcomes of this research demonstrate the effectiveness of a secure SSDLC in developing resilient software systems and underscore the importance of continuous security education for both developers and educators. The insights gained from this project contribute to the field of secure software development and offer valuable guidance for educators and developers engaged in similar initiatives.


\section{Introduction} % The section number is displayed
The increasing prevalence of cyberattacks in today's digital landscape presents significant challenges for educational institutions, particularly those that lack comprehensive cybersecurity measures [45]. High schools, which often handle sensitive student data and manage critical educational resources, are particularly vulnerable to threats such as spoofing, identity theft, social engineering, and unauthorized access [1]. Despite this, there is a notable gap in cybersecurity awareness and education among high school educators, who play a pivotal role in shaping students' understanding of technology and its applications.
The primary motivation behind this experimental educational research project is to bridge this gap by equipping high school educators with the knowledge and skills necessary to understand and implement cybersecurity practices within their schools. By exposing educators to the intricacies of cybersecurity and the secure software development process, this project seeks to foster a deeper understanding of how school management systems operate and the potential vulnerabilities they face [5].
Central to this initiative is the Secure Software Development Life Cycle (SSDLC), a methodology that emphasizes the integration of security considerations at every stage of the software development process [3]. Through a project-based learning approach, the research team, led by a collegiate computer science student, developed a prototype school management system, simulating a typical high school database that stores and manages student account information. This system serves as a practical example to demonstrate the importance of secure software development practices.
The project is structured to guide educators through the various phases of the SSDLC, including requirements gathering, design, implementation, testing, and maintenance, with a particular focus on identifying and mitigating security vulnerabilities [2]. Educators are not only introduced to technical concepts such as user authentication, data encryption, access control, and threat modeling but they are also actively involved in the process of attacking and defending the system. This hands-on experience is designed to help them recognize potential security threats, understand how cyberattacks are executed, and learn how to implement effective countermeasures [3].
The insights gained from this project underscore the critical need for continuous education in secure software development and cybersecurity. As cyber threats continue to evolve, so too must the strategies and practices employed to protect educational institutions. This research provides valuable guidelines for educators and developers alike, emphasizing the importance of a proactive, security-first approach in software development and system maintenance [6].
 \cite{1}

\subsection{Subsection} % Subsections are also numbered
This is a subsection.

\section{Methodology}
In this research, a project-based learning approach was employed to develop a miniature high school application system, designed to simulate a high school database managing student account information. This methodology facilitated the practical application of secure software development principles and provided a framework for identifying and addressing common security vulnerabilities using the STRIDE threat modeling methodology.

\subsection{Reserach Phase}
The primary focus is on the Secure Software Development Life Cycle (SSDLC). SSDLC is a process that incorporates security considerations at every stage of software development. It aims to identify and address security issues early, ensuring the final product is secure and robust. With this project, we have developed a school information database following a secure SSDLC process. The database created contains relevant information for students, teachers, and parents/guardians. The learning objectives are to understand the technical aspects of a school system and to equip staff members with the knowledge to recognize and prevent security threats.
Planning Phase 
During the planning phase of the Software Development process, we identified the type of application to be developed, brainstormed on components and requirements for the system, and evaluated the resources (people) and timeframe available for full completion of the project. 
Two types of diagrams were drawn to visually present the system, the Entity Relationship Diagram (ERD) and the Data Flow Diagram (DFD). In the ERD, all components of the system are shown with entities (objects), attributes (properties), and relationships between them. The ERD served as a foundation for building our database. It was the design upon which we planned to build our database. The CSV file (spreadsheet) was populated per the ERD diagram to link the entities, and attributes and show the relationships. 
The second diagram, the DFD, visualized a flow of information in the system. Each line shows how the data moves from one component to another and has been analyzed for potential cybersecurity risks and vulnerabilities. That became in our case an opportunity to model specific threads.  We developed a Threat Model of how our application system will interact with different components based on the STRIDE framework (Spoofing, Tampering, Repudiation, Information disclosure (privacy breach or data leak), Denial of service, and Elevation of privilege. In addition, we identified an opportunity to model SQL injection and Cross-Site Scripting.
\subsection{Development Phase}
During the development phase, our primary objective was to create a secure and functional school management system while deliberately introducing vulnerabilities for testing purposes. This phase encompassed the establishment of a robust database and the development of the website’s front-end and back-end components. The database was set up on SureServer, with a focus on secure authentication and student information management. The design process involved creating an Entity Relationship Diagram (ERD) to model entities such as Student, Teacher, Course, Section, and Enrollment, and defining their relationships. Data segregation practices were employed, including the duplication of the database to facilitate collaborative development without cross-impact among team members.

\subsection{Attack Phase}
The attack phase of this project was designed to demonstrate how an attacker could exploit vulnerabilities to gain unauthorized access to the school system. We employed techniques aligned with the STRIDE threat modeling framework and we explored potential vulnerabilities through cross-site scripting (XSS), escalation of privilege (EoP), and SQL injection attacks.
Our analysis focused on identifying key security weaknesses, such as improper input sanitization, lack of hashing, exposure of sensitive data in plaintext, insufficient data verification, and flawed implementation of role-based access control. We also examined scenarios where user inputs are directly reflected in the HTTP response, which can be exploited in XSS attacks. These vulnerabilities serve as entry points for attackers to gain unauthorized access to the school system, highlighting the critical need for robust security measures in web applications.

\subsection{Maintenance Phase}
The Maintenance phase in the Secure Software Development Life Cycle (SSDLC) is critical for sustaining the security, functionality, and overall integrity of the software after deployment. This phase is not just about fixing bugs but also about ensuring that the system adapts to evolving threats, maintains compliance with security standards and regulations, and continues to perform securely under ever-changing conditions. Key areas of focus include continuous monitoring, patch management, user education, data backup, and risk management. 
Continuous monitoring will assist in detecting security incidents before they escalate to significant threats. Implementing a Security Information and Event Management (SIEM) system such as Splunk1 or IBMQRadar2, enables real-time analysis of security alerts that are generated by application and network hardware. This system aggregates and correlates log data, providing insights into potential security breaches. Automated alerts should be configured to notify administrators of suspicious activities, such as unusual login patterns or unauthorized access attempts, such as our attempts at SQL injection during the testing phase [23]. 
Regular security assessments, including vulnerability scanning and penetration testing, are extremely important for identifying and mitigating new vulnerabilities in the system. Tools like OWASP ZAP3 (Zed Attack Proxy) and Burp Suite are commonly used for these purposes. Vulnerability scans should be performed regularly to detect weaknesses such as outdated software versions, insecure configurations, and missing patches [25].
Patch management is crucial for maintaining system security. Unpatched software is one of the most common entry points for attackers. According to the SANS Institute4, timely application of patches can prevent up to 90% of cyber-attacks [26]. Organizations should follow a structured patch management process, including testing patches in a controlled environment before deployment to ensure they do not introduce new issues.
Even the most secure systems can be compromised by human error. Social engineering attacks, such as phishing, remain a significant threat, exploiting human psychology rather than technical vulnerabilities. Regular training sessions should be conducted to educate users on recognizing phishing emails, avoiding suspicious links, and using strong, unique passwords. Security awareness training platforms, such as KnowBe45 or SANS Security Awareness, can be leveraged to simulate phishing attacks and provide real-time feedback to users [28]. This continuous education helps foster a culture of security within the organization, reducing the likelihood of successful social engineering attacks. Although our project did not include a social engineering attack due to time constraints, this would be an easy way to gain entrance into our database [6].
Data integrity and availability are crucial for the resilience of the school management system. Regular backups should be performed using robust, automated tools such as Veeam6 or Acronis7, ensuring that data is backed up to both local and off-site locations. This dual-location strategy protects against both cyber-attacks and physical disasters.
Backups should be encrypted to protect data from being compromised during storage or transit. The Advanced Encryption Standard (AES-256) is recommended for its strength and reliability [8]. Regularly testing the restoration process is also important to ensure that data can be recovered quickly and securely in the event of a breach or data loss [29].
The threat landscape is constantly evolving, with new vulnerabilities and attack vectors emerging regularly. Staying informed about these threats is essential for maintaining system security. The use of threat intelligence platforms, such as Mandiant Threat Intelligence8 or Recorded Future9, can provide real-time insights into emerging threats and vulnerabilities [31].
Threat modeling should not be a one-time activity but an integral part of the maintenance phase. As the system evolves, new features, integrations, or changes in architecture may introduce new risks. Regularly updating the threat model helps in identifying potential vulnerabilities that could be exploited by attackers.
Tools such as Microsoft Threat Modeling Tool10 or OWASP Threat Dragon11 can assist in visualizing the system architecture, identifying potential threats, and determining the effectiveness of current security measures [32]. Risk management strategies should be revisited regularly to ensure that any new or emerging risks are addressed promptly [33].

\subsection{RESEARCH}
The research conducted in this study focused on identifying and exploiting common security vulnerabilities in a PHP-based web application [39]. Specifically, the study explored Cross-Site Scripting (XSS), Spoofing, Elevation of Privilege (EoP) through Parameter Tampering and Information Disclosure attacks. To demonstrate these vulnerabilities, the application was intentionally modified to weaken its security posture, such as removing input sanitization, bypassing role-based access controls, and exposing sensitive information. The attacks were carried out by manipulating client-side input, such as altering hidden form fields to escalate privileges or injecting malicious scripts to execute in other users' browsers. Through these controlled experiments, we were able to illustrate how these vulnerabilities can be exploited in real-world scenarios and emphasize the importance of implementing robust security practices to prevent such attacks.
Cross-Site Scripting (XSS)
XSS vulnerabilities arise when an attacker injects malicious scripts into web pages viewed by other users. In our study, the vulnerability was demonstrated by removing input sanitization and output encoding, allowing for the injection of malicious scripts [33]. For example, a user could input <script>alert('XSS');</script>, which would be executed in the browser of anyone viewing the page. To prevent XSS attacks, it is crucial to implement proper input sanitization and output encoding. This involves using functions like htmlspecialchars() to encode special characters before displaying user inputs. Additionally, validating input formats and implementing a Content Security Policy (CSP) can further mitigate XSS risks.

\subsection{Spoofing}
Spoofing involves falsifying data or identities to deceive users or systems. In the case study, spoofing vulnerabilities were introduced by removing protections that verify user inputs. This allowed attackers to forge requests and potentially gain unauthorized access by manipulating data such as user levels. To prevent spoofing, it is essential to validate all user inputs against expected formats and implement mechanisms like CSRF tokens to verify the legitimacy of requests [34]. Regularly updating and auditing security measures also help in mitigating spoofing risks.
\subsection{SQL Injection}
SQL Injection occurs when user input is executed as part of an SQL query, potentially allowing attackers to manipulate the database. In our application, this was prevented by using prepared statements and parameterized queries. Using prepared statements and parameterized queries ensures that user inputs are treated as data rather than executable code. This practice protects against SQL injection by separating SQL logic from data [46].

\subsection{Tampering and Elevation of Privilege}
Elevation of Privilege (EoP), also known as privilege escalation, occurs when an attacker gains higher access rights than initially intended, allowing them to perform actions typically restricted, such as accessing sensitive information, executing commands, or altering system configurations. EoP attacks exploit vulnerabilities in software, operating systems, or applications to escalate privileges from a lower level (such as a regular user) to a higher level (such as an administrator or root user).
There are two primary types of privilege escalation: Vertical and Horizontal. In Vertical Privilege Escalation, an attacker elevates their privileges from a lower level to a higher level within the same system, such as a regular user gaining administrative privileges. In a Horizontal Privilege Escalation, an attacker accesses the privileges of another user with the same privilege level, often by exploiting vulnerabilities or stealing credentials [40].
Tampering involves unauthorized modifications to data, compromising its integrity. In our research, we identified tampering through data injection, modification, deletion, and replay attacks. These attacks can result in data corruption, fraud, or unauthorized access, leading to severe consequences, including financial loss, reputational damage, and operational disruptions [38].
Areas that we researched as a team but did not test due to time constraints were Information Disclosure, Repudiation, and Denial of Service (DoS).
Repudiation attacks occur when an attacker denies performing an action, making it difficult to trace activities. This can undermine transaction authenticity and integrity.  Implementing digital signatures, comprehensive audit logs, and two-factor authentication (2FA) can mitigate repudiation attacks. Encryption and strict access controls further enhance accountability and traceability.
Denial of Service attacks overwhelm a system, making it unavailable to legitimate users. This can be achieved by flooding the server with excessive requests. To prevent DoS attacks, implement rate limiting, use load balancing, and deploy network security solutions like firewalls and intrusion detection systems.
In addition to the Repudiation and Denial of Service attacks, our research also identified the potential risk of an Information Disclosure Attack. While we did not attempt this specific attack, it is important to understand the implications of such vulnerabilities. An information disclosure attack occurs when sensitive information is unintentionally exposed to unauthorized users. In web applications, this can happen due to improper error handling, inadvertent data exposure in responses, or poorly configured access controls [33].
For example, if such an attack were launched, one might observe that detailed error messages containing stack traces, database queries, or even user credentials are being returned to the client during certain operations. These disclosures could provide attackers with critical information about the system's architecture, underlying technologies, or expose vulnerabilities that could be exploited in subsequent attacks [46].
\subsection{DIAGRAMS}
\subsection{Entity Relationship Diagram (ERD)}
During the planning phase, we developed an Entity Relationship Diagram (ERD) to model the school database's structure. The ERD is organized into three main components: entities, attributes, and relationships. Entities represent the objects that the database tracks, such as Teacher, Course, Section, Enrollment, Student, Guardian, and Transcript. Attributes are the specific properties or characteristics of these entities; for instance, the "Student" entity includes attributes like FirstName, LastName, gradeLevel, SpecialtyProgram, GuardianID, and GuardianID2.
Relationships define how these entities interact with each other, depicted in the diagram by lines connecting them. These lines include cardinality notations, which describe the numerical context of the relationships, indicating whether they are one-to-one, one-to-many, or many-to-many. For example, the relationship between Teacher and Section is one-to-many, as one teacher can instruct multiple sections. Similarly, the relationship between Transcript and Student is one-to-one, as each transcript corresponds to a single student. The ERD visually represents these relationships, making it clear how data within the school database is interconnected. 

\subsection{Data Flow Diagram (DFD)}
Data Flow Diagram (DFD) is a visual way to represent the flow of information within a system. DFDs help to spot inefficiencies in the system (in our case database) and find opportunities to improve functionality [50]. In this project, the DFD illustrates the flow of information between key components, including the Database, Account Login, Check Privileges, User, and Home Page. 
For instance, when a user attempts to log in, their input is sent to the Account Login process, where it is verified against the database. The Check Privileges process then assesses the user's permissions, determining the appropriate response. Depending on the result, the system either grants access to the Home Page, where the user can view and manage their information, or denies access, ensuring that only authorized users can edit student records. This flow of information is crucial in maintaining the integrity and security of the system while providing users with the functionality they require.
DATABASE & WEB APPLICATION
During the development phase, our primary objective was to create a secure and functional school management system while deliberately introducing vulnerabilities for testing purposes. This phase encompassed the establishment of a robust database and the development of the website’s front-end and back-end components. 
We utilized SureServer's11 hosting services (https://cp.s425.sureserver.com/) to establish the necessary databases for authentication and student information management. The initial step involved creating a secure and robust database structure to safeguard the integrity and confidentiality of user data. To facilitate collaborative development and testing, the original databases were duplicated and shared among team members, ensuring that individual modifications did not interfere with each other's work.
The code establishes the front-end interface for user authentication. It includes a login form where users input their username and password and select their role (student or teacher). The form submission triggers the login.php script, which processes the credentials against the database.
The validation script ensures that both username and password fields are populated before submission, reducing the risk of empty input fields causing unnecessary processing on the server side.
Initially, the databases and code were securely designed to prevent common vulnerabilities. However, to demonstrate the importance of security, the code was later modified to introduce intentional vulnerabilities. These modifications illustrated how poor security practices could make the system vulnerable to attacks, such as Cross-Site Scripting (XSS) and spoofing. This phase of the project highlighted the critical need for secure coding practices in the development of web applications, emphasizing the impact of lax security measures on the overall system integrity.
The website development involved designing the front-end interface with HTML and implementing server-side logic with PHP. The login system, central to the user authentication process, was constructed with form validation to ensure proper credential submission. Intentional vulnerabilities were introduced to highlight common security issues. For instance, Cross-Site Scripting (XSS) vulnerabilities were demonstrated by removing input sanitization, allowing for malicious script injection. Spoofing vulnerabilities were showcased by bypassing role-based access controls, illustrating how unauthorized data manipulation could occur. SQL injection was attempted, although modern security measures mitigated its impact, demonstrating the need for prepared statements and parameterized queries
\subsection{ATTACKS/CHALLENGES}
The secure development of software is an ongoing battle against a constantly evolving landscape of threats and vulnerabilities. Throughout the development of the school management system, numerous security challenges were encountered, each requiring a unique approach for identification, mitigation, and resolution. These challenges were integral to the learning process, as they underscored the necessity of integrating security at every phase of the Secure Software Development Life Cycle (SSDLC).
In this section, we delve into the various attacks and security vulnerabilities that were identified and exploited during the project's testing phase. We examine the methodologies used to simulate these attacks, the challenges faced in mitigating them. The attacks discussed include spoofing, SQL injection, tampering, and

\subsection{Figure 1. Entity Relationship Diagram (ERD)}

\subsection{Figure 2. Data Flow Diagram (DFD)}

elevation of privilege (EoP), each of which exposed different facets of the system’s security.
Results
Spoofing, for example, was a significant challenge that involved the falsification of data and identities to deceive the system. This vulnerability was introduced by removing protections that verify user inputs, leading to potential unauthorized access through manipulated data. This attack highlights the importance of validating user inputs and implementing mechanisms like CSRF tokens to prevent such attacks. Regular updates and audits of security measures were also identified as critical in mitigating spoofing risks.
To demonstrate the impact of vertical privilege escalation, we executed a parameter tampering attack targeting the user registration process of a web application. The objective was to exploit a vulnerability in the signup.php form, responsible for creating new student accounts, to create an account with elevated privileges. Specifically, we aimed to assign a "teacher" role to an account that should have been restricted to a "student" role.
Using browser developer tools, we (the attacker) manipulated a hidden input field labeled userLevel within the form. Initially, this field was set to assign the "student" role to new accounts. However, by altering its value from "student" to "teacher" before submitting the form, the attacker tricked the server into creating an account with "teacher" level privileges instead of the intended "student" level. As a result, we successfully escalated our privileges within the system, gaining unauthorized access to features and data reserved for users with a "teacher" role.
This attack highlights the significant risks associated with trusting client-side input for critical security decisions, such as role assignments [49]. The vulnerability could have been mitigated by enforcing role-based access control (RBAC) on the server side, ensuring that role assignments and other sensitive data are validated by the server, rather than relying on potentially manipulated client-side input. Proper server-side validation and the avoidance of using hidden fields for security-critical information are essential steps in protecting against such privilege escalation attacks [10].
To illustrate this security vulnerability, we conducted an SQL injection attack using payloads such as {' OR 1=1--} and {' OR '1' = '1'} for both the username and password fields. These inputs are designed to manipulate the SQL query to always return true, potentially granting unauthorized access. Contrary to expectations, these attempts were thwarted, resulting in a mysql_sql_exception error. Upon further investigation, it was revealed that while the code contained vulnerabilities allowing user input to escape the query statement using quotation marks, the attack was unsuccessful due to robust security measures inherent in the MySQL database. Specifically, the database’s input validation mechanisms intercepted and neutralized the attempted manipulation, preventing the SQL query from being compromised [36].
\subsection{Challenges}
We encountered several challenges that impacted our ability to successfully exploit certain vulnerabilities. For instance, while our PHP code was deliberately structured to be vulnerable to SQL injection, the up-to-date database system in use presented significant obstacles. Modern database servers incorporate built-in security mechanisms that mitigate the risk of SQL injection attacks, making it difficult to successfully inject malicious code[47]. As a result, to effectively demonstrate the vulnerability, we would have needed to downgrade to an older version of the database server, one lacking these advanced security features. 
This experience underscored the essential importance of secure coding practices, regular software updates, and comprehensive security measures in protecting web applications from diverse threats. The challenges encountered not only strengthened the security of the school management system but also offered critical insights into best practices for secure software development. These experiences reveal the ever-evolving nature of cybersecurity and emphasize the need for ongoing vigilance, continuous education, and adaptable strategies to effectively counter emerging threats.
\subsection{PREVENTION AND MAINTENANCE}
In our research and application, we rigorously examined a range of security vulnerabilities and devised methodologies to mitigate these risks effectively. This section outlines the prevention techniques recommended to address vulnerabilities identified, including SQL Injection, Cross-Site Scripting (XSS), Elevation of Privilege (EoP), Spoofing, Parameter Tampering, and Principles of Least Privilege (PoLP).
Input Sanitization and Output Encoding for XSS Prevention
To prevent XSS attacks, all user inputs must be sanitized and encoded before they are displayed on web pages. Utilizing functions such as htmlspecialchars() in PHP ensures that any HTML tags or scripts are treated as plain text, thus neutralizing potential XSS threats. Additionally, implementing Content Security Policy (CSP) headers further mitigates XSS risks by controlling the sources from which scripts can be executed [36].
Role-Based Access Control (RBAC) and Server-Side Validation for Spoofing and EoP Prevention
Critical security decisions, such as role assignments, should never rely on client-side input. Instead, roles must be enforced and validated server-side using trusted data. Robust server-side validation is essential to verify user permissions before granting access to restricted resources or actions, thereby preventing spoofing and privilege escalation attacks [44].
Parameter Tampering Mitigation for Vertical Privilege Escalation Prevention
To safeguard against parameter tampering, it is crucial to avoid using hidden fields for sensitive information, such as user roles. Instead, roles and other critical attributes should be determined through server-side logic. Validating all incoming requests and employing security tokens or checksums can ensure that form data remains unaltered during transmission [47].
Error Handling and Information Disclosure Protection
Error handling mechanisms should be configured to display generic error messages, thus preventing the unintentional exposure of sensitive information such as stack traces or database queries. Detailed error logs should be securely stored on the server and accessible only to authorized personnel. Moreover, sensitive data should be excluded from URLs, and access to system files or directories should be restricted to prevent unauthorized access [42].
Cross-Site Request Forgery (CSRF) Protection
To counter CSRF attacks that exploit authenticated sessions to perform unauthorized actions, CSRF tokens should be implemented in forms and verified by the server for any state-changing requests. This practice ensures that only legitimate requests are processed [34].
Prepared Statements for SQL Injection Prevention
Although our primary focus was on XSS and spoofing, SQL injection remains a critical risk. Consistently using prepared statements for database interactions can effectively prevent SQL injection attacks by ensuring that user inputs are properly escaped and parameterized [41].
HTTPS and Secure Communication
Ensuring that all data transmitted between the client and server is encrypted using HTTPS is paramount. This measure prevents attackers from intercepting and manipulating data in transit, which could otherwise be exploited in attacks such as parameter tampering or spoofing [7].
Security Audits and Regular Monitoring
Regular security audits are essential to proactively identify and address vulnerabilities. The use of security tools and monitoring systems can detect and respond to suspicious activities, such as unauthorized access attempts or anomalies in user behavior, thereby enhancing the overall security posture [48]. Regular security audits are essential to proactively identify and address vulnerabilities. The use of security tools and monitoring systems can detect and respond to suspicious activities, such as unauthorized access attempts or anomalies in user behavior, thereby enhancing the overall security posture [48].
Principle of Least Privilege (PoLP)
Adhering to the Principle of Least Privilege ensures that users and systems are granted only the minimum privileges necessary to perform their tasks. This approach reduces the risk of privilege escalation and minimizes the potential impact of security breaches [8].
By implementing these comprehensive prevention strategies, developers can significantly reduce the risk of security vulnerabilities in PHP web applications. These measures not only protect against the specific attacks identified in this study but also contribute to a broader security framework, guarding against a wide array of potential threats.
\subsection{CONCLUSION}
In summary, this study has effectively demonstrated the integration of secure software development principles within an educational framework through a project-based learning approach. The research underscored the importance of embedding security practices throughout the Secure Software Development Life Cycle (SSDLC), highlighting their role in fortifying the integrity of a school management system. By meticulously applying SSDLC methodologies, the project team identified and mitigated critical vulnerabilities such as Cross-Site Scripting (XSS), spoofing, and Elevation of Privilege (EoP), thereby enhancing the overall resilience of the software.
The findings emphasize the efficacy of a structured SSDLC in developing secure applications and the significant benefits of continuous security education for both educators and developers. The practical insights gained from this study contribute valuable knowledge to the field of secure software development and offer a replicable framework for educators to incorporate these principles into their teaching practices. As digital systems become increasingly integral to education, ensuring their security will be paramount in protecting sensitive information and maintaining the trust of users [45].
However, the project encountered several challenges, including resource limitations and the complexity of communicating advanced security concepts to non-specialist audiences. Future research should aim to refine educational methodologies, explore additional security measures, and broaden the scope to encompass diverse educational environments. As digital systems increasingly underpin educational operations, this project highlights the imperative of adopting robust security practices to protect sensitive data and maintain user trust, advocating for a proactive approach to security awareness in academic settings.
\subsection{REFERENCES}
[1] National Cyber Security Alliance. (2021). Cybersecurity in Schools: Protecting Our Future. Retrieved from https://staysafeonline.org/
[2] Microsoft. (2021). Security Development Lifecycle (SDL) Practices. Retrieved from https://docs.microsoft.com/en-us/security/engineering/security-development-lifecycle
[3] HackerOne. (n.d.). What is SSDLC (Secure Software Development Life Cycle) Retrieved from https://www.hackerone.com/knowledge-center/what-ssdlc-secure-software-development-life-cycle
[4] U.S. Department of Education. (2020). Protecting Student Privacy: Cybersecurity Considerations for K-12 Schools and School Districts. Retrieved from https://studentprivacy.ed.gov/resources/cybersecurity-considerations-k-12-schools-and-school-districts
[5] Ponemon Institute. (2021). The Importance of Cybersecurity Awareness Training for Employees. Retrieved from https://www.ponemon.org/library/importance-of-cybersecurity-awareness-training
[6] Mitnick, Kevin. D., & Simon, William. L. (2011). *The Art of Deception: Controlling the Human Element of Security.* John Wiley & Sons.
[7] National Institute of Standards and Technology. (2022). NIST Special Publication 800-175B Revision 1: Guide to Cyber Threat Information Sharing. Retrieved from https://nvlpubs.nist.gov/nistpubs/SpecialPublications/NIST.SP.800-175Br1.pdf
[8] Schneier, Bruce. (1996). *Applied Cryptography.* Retrieved from Schneier on Security. Retrieved from https://mrajacse.wordpress.com/wp-content/uploads/2012/01/applied-cryptography-2nd-ed-b-schneier.pdf
[9] Microsoft. (2021). Security Best Practices. Retrieved from Microsoft Documetation
[10] Open Web Application Security Project (OWASP) Foundation. (2021). Role Based Access Control. Retrieved from OWASP Foundation. (n.d.). Access Control. Retrieved from https://top10proactive.owasp.org/v4/en/c1-accesscontrol
[11] Amazon Web Services. (2021). Data Encryption. Retrieved from AWS Documentation
[12] Rakib, Rashaduzzanan. (2021). What is a database vulnerability? What is a database attack? Medium. Retrieved from https://medium.com/@rakib.ssa/what-is-a-database-vulnerability-what-is-a-database-attack-a098956474ef#:~:text=SQL%20Injection%20Attacks%3A%20These%20attacks,database%2C%20often%20using%20automated%20tools
[13]SalvationData. (n.d.). 6 types of database hacks used to obtain unauthorized access. Retrieved from https://www.salvationdata.com/crime-cases/6-types-of-database-hacks-use-to-obtain-unauthorized-access/
[14] Menezes, Alfred. J., van Oorschot, Paul. C., & Vanstone, Scott. A. (1996). *Handbook of Applied Cryptography.* CRC Press. Retrieved from https://cacr.uwaterloo.ca/hac/
[15] Security Intelligence. (Retrieved 2024). What is STRIDE threat modeling? Anticipate cyberattacks. Retrieved from https://securityintelligence.com/articles/what-is-stride-threat-modeling-anticipate-cyberattacks/
[16] Cyber Insight. (Retrieved 2024). What is STRIDE in cyber security? Retrieved from https://cyberinsight.co/what-is-stride-in-cyber-security/
[17] Microsoft Learn. (Retrieved 2024). Threat modeling tool threats. Retrieved from https://learn.microsoft.com/en-us/azure/security/develop/threat-modeling-tool-threats
[18] National Institute of Standards and Technology. (2021). Guide to Secure Web Services. Retrieved from NIST
[19] Open Web Application Security Project (OWASP) Foundation. (2021). Access Control. Retrieved from OWASP
[20] Microsoft. (2021). Security Development Lifecycle (SDL) Practices. Retrieved from Microsoft Docs
[21] Amazon Web Services. (2021). Best Practices for Security, Identity, & Compliance. Retrieved from AWS Documentation
[22] SANS Institute. (202). Critical Security Controls. Retrieved from SANS
[23] Splunk. (2023). SIEM: Security Information and Event Management. Retrieved from https://www.splunk.com
[24] National Institute of Standards and Technology. (2012). Computer Security Incident Handling Guide (Special Publication 800-61, Revision 2). Retrieved from https://nvlpubs.nist.gov
[25] Open Web Application Security Project (OWASP) ZAP. (2023). OWASP Zed Attack Proxy Project. Retrieved from https://www.owasp.org
[26] SANS Institute. (2020). Patch and Vulnerability Management. Retrieved from https://www.sans.org
[27] Mitnick, Kevin., & Simon, William. (2002). *The Art of Deception: Controlling the Human Element of Security.* Wiley.
[28] KnowBe4. (2023). Security Awareness Training. Retrieved from https://www.knowbe4.com
[29] Veeam. (2023). Backup Solutions. Retrieved from https://www.veeam.com
[30] National Institute of Standards and Technology. (2001). Advanced Encryption Standard (AES). Retrieved from https://www.nist.gov
[31] Mandiant. (2023). Threat Intelligence. Retrieved from https://www.mandiant.com
[32] Microsoft. (2023). Microsoft Threat Modeling Tool. Retrieved from https://www.microsoft.com
[33] Open Web Application Security Project (OWASP). (n.d.). Cross-Site Scripting (XSS). Retrieved from https://owasp.org/www-community/attacks/xss/
[34] Open Web Application Security Project (OWASP). (n.d.). Cross-Site Request Forgery (CSRF). Retrieved from https://owasp.org/www-community/attacks/csrf
[35] PHP Manual. (n.d.). htmlspecialchars. Retrieved from https://www.php.net/manual/en/function.htmlspecialchars.php
[36] Halfond, William. G., Viegas, Jeremy., & Orso, Alessandro. (2006). A classification of SQL injection attacks and countermeasures. In *Proceedings of the IEEE International Symposium on Secure Software Engineering. Retrieved from https://api.semanticscholar.org/CorpusID:5969227
[37] Mitnick, Kevin. (2002). Social Engineering Attack Demonstration. Retrieved from https://www.youtube.com/watch?v=xsg9BDiwiJE
[38] Open Web Application Security Project (OWASP) (n.d.). Tampering. Retrieved from https://owasp.org/www-community/attacks/Tampering.
[39] Cybersecurity and Infrastructure Security Agency. (2023). AA23-215A: Guidance on Addressing Recent Exploitation of Accellion File Transfer Appliance Vulnerabilities. Retrieved from https://www.cisa.gov/news-events/cybersecurity-advisories/aa23-215a
[40] Open Web Application Security Project (OWASP). (n.d.). Elevation of Privilege. Retrieved from https://owasp.org/www-community/attacks/Elevation_of_Privilege.
[41] National Institute of Standards and Technology(NIST). (2008). NIST Special Publication 800-95: Guide to Secure Web Services. Retrieved from https://nvlpubs.nist.gov/nistpubs/Legacy/SP/nistspecialpublication800-95.pdf
[42] National Institute of Standards and Technology (NIST). (2018). Special Publication 800-61 Revision 2: Computer Security Incident Handling Guide. Retrieved from https://nvlpubs.nist.gov/nistpubs/Legacy/SP/nistspecialpublication800-61r2.pdf
[43] National Institute of Standards and Technology (NIST). (2020). Special Publication 800-53 Revision 5: Security and Privacy Controls for Information Systems and Organizations. Retrieved from https://nvlpubs.nist.gov/nistpubs/SpecialPublications/NIST.SP.800-53r5.pdf
[44] Open Web Application Security Project (OWASP). (2024). OWASP Cheat Sheet Series: Spoofing Prevention. Retrieved from https://cheatsheetseries.owasp.org/cheatsheets/Spoofing_Prevention_Cheat_Sheet.html
[45] CapRadio. (2024). One reason school cyberattacks are on the rise: Schools are easy targets for hackers. Retrieved from https://www.capradio.org/articles/2024/03/11/one-reason-school-cyberattacks-are-on-the-rise-schools-are-easy-targets-for-hackers/#:~:text=months%20to%20recover.-,Schools%20are%20easy%20targets%20for%20hackers,cyber%20security%20experts%20on%20staff.%22
[46] Open Web Application Security Project (OWASP). (2024). OWASP Cheat Sheet Series: Information Disclosure. Retrieved from https://cheatsheetseries.owasp.org/cheatsheets/Information_Disclosure_Cheat_Sheet.html
[47] MySQL. (2010). New MySQL Enterprise Firewall Prevents SQL Injection Attacks. Retrieved from https://dev.mysql.com/blog-archive/new-mysql-enterprise-firewall-prevent-sql-injection-attacks/
[48] Cybersecurity and Infrastructure Security Agency (CISA). (2023). AA23-215A: Emerging Cyber Threats and Mitigation Strategies. Retrieved from https://www.cisa.gov/news-events/cybersecurity-advisories/aa23-215a
[49] OWASP Foundation. (2024). OWASP Top 10 Client-Side Security Risks. Retrieved from https://owasp.org/www-project-top-10-client-side-security-risks
[50] UC Irvine. 2024. Data Flow Diagram Risk Assessment Program. Retrieved from https://www.security.uci.edu/program/risk-assessment/data-flow-diagram/




\begin{thebibliography}{9}
\bibitem{1}
    % make an example citation
    Doe, J. (2020). Lorem Ipsum. \textit{Journal of Lorem Ipsum}, 1(1), 1-10.
\end{thebibliography}

\end{document}